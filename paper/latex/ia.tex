\documentclass[12pt]{turabian-researchpaper}



\usepackage[utf8]{inputenc}
\usepackage{turabian-formatting}
\usepackage[backend=biber]{biblatex-chicago}
\usepackage{indentfirst}



\addbibresource{mybib.bib}



\begin{document}
\title{To what extent was the Yishuv's policy in the Arab Israeli War of 1948 one of territorial expansion and expulsion of the native Arab population?}



\author{Will Kaufman \\ Mr. Niedringhaus \\ IB World History}
\maketitle



\section{Section A: Plan of Investigation}
In this investigation, several methodologies will be utilized to thoroughly examine the question of whether the Yishuv fought the Arab Israeli War of 1948 for territorial expansion and expulsion of the native Arab population.  The primary method of investigation will be the analysis of conflicting historiographies, generally taken from Israel's perspective and the Arabs' perspective.  In order to ensure a comprehensive analysis of these competing historiographies, a combination of primary and secondary sources sources will be used to weigh the various intentions of both parties involved in the war and the responses elicited from these actions.  For secondary sources, the date of publication will be considered, as, in accordance with the ``thirty-year rule" in both the United Kingdom and the state of Israel, secondary sources published before 1978 will not take into account information that was declassified after 1978.

%=================Section B====================

\section{Section B: Summary of Evidence}
The Arab-Israeli War of 1948 was divided into two different parts: the civil war fought between the Jewish state and the Palestinian Arabs, and the second phase of the war fought between the Jewish state and other Arab nations.  The first phase of the war began shortly after the United Nations (UN) General Assembly vote on Resolution 181 on 29 November 1947.
\footcite[][]{tal}
The UN Special Committee on Palestine (UNSCOP) worked from April to September of 1947 to research and identify a solution to the Jewish-Arab conflicts, and to identify an effective solution to the transition from British mandatory rule to autonomous rule.  The committee ultimately proposed the partition of the Palestinian territory to form two nations, one for the creation of a Jewish state and one for the Arab population.
\footcite[][22]{pappe}
The partition resolution (Resolution 181) conflicted with the demands from the Arab Higher Committee and the Arab League, both of which opposed the partition plan and instead called for a single Palestinian state.
\footcite[][23]{pappe}
The Arab Higher Committee furthered its demands by requesting that Jewish immigration into Palestine be severely limited.  Despite the dissent from the various Arab representatives, and with the help of the Jewish Agency's far more diplomatic approach to the UNSCOP, the partition resolution passed.  The next day, on November 30, Palestinian Arabs began perpetrating sporadic acts of violence in urban centers as well as Jewish settlements.
\footcite[][77]{morris}
By December 1947, acts of retaliation from both sides escalated the conflict into a civil war.



Before the start of the civil war, it was unclear whether the Arab Palestinians or the Yishuv (the Jewish community in Palestine before the creation of Israel) had the upper hand in battle.  The Palestinians outnumbered the Yishuv by a factor of two (1.2 million Palestinians as compared to 630,000 Jews), and the Palestinians controlled far more land and more of the highland areas than did the Jewish community.
\footcite[][30]{bartal}
However, the Hagana, the Jewish military organization, was far better equipped, better trained, and had more military experience than the Palestinian Arab militias and the Arab Liberation Army (ALA) that fought in the war.
\footcite[][81]{morris}
This question of military superiority caused other nations, including the United States, to abstain from supporting the creation of the Jewish state.



Initially, the war went well for the Palestinians.  Fawzi al-Qawqji, field commander in the ALA, led several successful operations against Jewish convoys in January of 1948, and attacks on Jewish kibbutzim by Arab militias were initially successful.
\footcite[][79]{pappe}
By fighting small, localized conflicts, the Palestinians (with outside Arab support) were able to win several battles, hold territory between Jewish urban centers and roads, and force the Jewish leadership to question their military strategy.  The Hagana in the early stages of the war conformed to a purely defensive strategy, working to protect Jewish urban centers as well as more isolated kibbutzim.  The early success for Palestinians was short-lived, as new military policies effectively defended Jewish settlements and opened the roads for convoys to enter urban centers.  However, the Hagana did not go on the offensive in the first months of the conflict, but instead defended against Arab attacks and issued orders of retaliation against Arab aggression.
\footcite[][68]{tal}



By mid-March of 1948, the inconclusive civil war between the Palestinians and the Jews, as well as the threat of an invasion by surrounding Arab countries in mid-May, forced the Jewish leadership to embrace a different military strategy.  David Ben Gurion, in accordance with members of the Hagana General Staff adopted Plan Dalet to transition to an offensive strategy and prepare for the coming invasion of Arab troops.  Plan Dalet called for the securement of all territory outlined in the UN partition for the Jewish state, as well as territories bordering the future belligerent Arab states and any ``enemy bases" in the Palestinian territory.
\footcite{khalidi}
By ``enemy bases," the plan implied any Palestinian villages that held Arab forces or that had been the base of a past enemy attack.
\footcite[][119]{morris}
Before implementing the offensive plan, the Hagana waited until the number of British troops diminished in observation of the UN Partition Resolution and the transition from Great Britian's mandatory state to autonomous rule.
\footcite[][32]{herzog}
Fewer British troops meant less resistance with the new offensive.  Furthermore, the Hagana received much needed Czech weapons and ammunition to fully implement Plan D.
\footcite[][116]{morris}



The first operation to implement Plan D was Operation Nahshon, in which Hagana troops went on the offensive from the Kibbutz Hulda to Jerusalem on April 4th 1948 in order to open the read to Jerusalem for supplies.
\footcite[][121]{morris}
Within only a few days, members of the Irgun and Lehi, underground Jewish organizations that fought against the British mandate before 1948, led a massacre on April 9th in the Arab village of Deir Yassin in which 100 men, women, and children were killed.
\footcite{nabka}
Until My 15th, when the surrounding Arab nations invaded and the second phase of the war began, the Hagana continued offensive operations in accordance with Plan Dalet.



As a result in the change in strategy in the effort to secure all of the Jewish settlements and transportation routes, the Hagana displaced hundreds of thousands of Palestinian Arabs, who fled to neighboring Arab countries.  By the end of the 1948 war, the state of Israel was created with an additional $5000km^2$ of land not included in the original UN partition resolution.
\footcite{lorch}
Over 600,000 Palestinian Arabs fled their homes to surrounding Arab countries, and over 850,000 Jews similarly fled from surrounding countries into Israel.
\footcite{bartal}

%=================Section C====================

\section{Section C: Evaluation of Sources}
\noindent \fullcite{morris}



Benny Morris' \citetitle{morris} is a book published in 2008.  The purpose of the book is to further the understanding of the first Arab-Israeli conflict through a revisionist lens by examining new documents released by Israeli, British, and U.S. sources in accordance with the ``30-year rule," as well as previously available sources.  The book provides a valuable presentation of both the Arab and the Jewish side of the war through primary sources from both parties, thus largely avoiding partisan bias on the part either the Arab or Jewish people.  However, the book is limited as an unbaised source because the author Benny Morris was born in Israel in the middle of the Arab-Israeli War of 1948 and self-identifies as a Zionist.  These facts may have influenced Morris to conciously select sources in support of Israel and the Zionist cause to fit his own personal beliefs.



~\\



\hangindent=\parindent \noindent \fullcite{nabka}



\citetitle{nabka} is an article written by the Institute for Middle East Understanding in May of 2013.  The purpose of the website is to ``to increase the public's understanding about the socio-economic, political and cultural aspects of Palestine, Palestinians and Palestinian Americans."  The purpose of the article specifically is to present various facts that support the ``ethnic cleansing" of the Palestinian territory in the late 1940's.  The source provides valuable perspectives from the Arab side of the Arab-Israeli War of 1948, outlines the atrocities committed by the Hagana, and emphasizes the extent to which the Yishuv actively expelled the native Palestinian population.  However, the source is limited because the Institute for Middle East Understanding is a pro-Palestinian organization which will bias the presentation of the facts in support of the Arabs.  In addition, the article selectively presents information that supports the Institute's own beliefs, thus failing to present the complete history of the 1948 war.

%=================Section D====================

\section{Section D: Analysis}



The Arab Israeli War of 1948 continued the conflict between the Jews and the surrounding Arabs, both within and outside of the Palestinian territory.  In the early stages of the war, nearly all perspectives concur on the Hagana's explicit defensive strategy against the Palestinian Arabs.
However, some sources indicate that the Yishuv changed strategies after the first months of war with the purpose of expelling the Arabs and ultimately controlling territory outside the allotted UN partition territory designated for Israel.  Other sources claim that the Hagana was forced, by nature of the war, to implement an offensive campaign in order to firmly establish transportation routes between urban centers, protect isolated Jewish settlements, and prepare for an impending invasion of Arab troops.
While the evidence available from the war corroborates the Hagana's change in strategy from defensive to offensive, it also indicates that the Hagana intended only to secure transportation routes, Jewish settlements, and prepare for other Arab countries to invade the territory, not to actively expel the Palestinian Arab population.  Extreme terrorist groups (such as LHI and IZL ) did commit horrendous acts against Arab populations, but these acts were ultimately insignificant in the mass exodus of Palestinian Arabs.  Furthermore, the lack of oranization and leadership on the part of the Palestinians catalyzed the disintigration of Palestinian society within Israel.



By the end of the year 1947, the UN partition resolution was released, and preliminary acts of violence were beginning in Palestine.  According to the Institute for Middle East Understanding, the Jewish leadership had no intention of following the UN partition plan and was instead planning to take over the Palestinian territory by force.
\footcite{nabka}
Jewish paramilitary groups such as IZL and LHI intensified their attacks against Arabs and British soldiers early on in the conflict, thus clearly displaying an unprovoked offensive strategy from the beginning of the war.
\footcite{nabka}
Already by December of 1947, only two months into the war, 75,000 Palestinian Arabs were displaced from their homes.



The statements above from the Institute for Middle East Understanding are not wholly accurate and do not reflect the entire situation during the war.  As claimed by Benny Morris, the Yishuv accepted the UN partition resolution and, for the first months of the war, held a purely defensive position against Arab attacks.
\footcite[][98-100]{morris}
Acts of Arab aggression were met with limited retaliation on the part of the Hagana, although other groups such as IZL and LHI were too difficult and extremist to control.
And while it is true that tens of thousands of Palestinians had left their homes by the end of 1947, it was less due to Israeli aggression than a host of other factors.  Most of the Palestinian leaders had fled the territory in late 1947, before the war had even begun, leaving the Palestinian people without sufficient organization and displaying a lack of faith in their own cause.  The Arab economy was debilitated from the start of the conflict, in part from the lack of organization and preparation surrounding economic policy (in contrast, the Yishuv had made ample preparations before the war to ready the economy for an armed conflict). And finally, the Palestinian Arabs, including most other people in both groups, initially believed that the conflict would only last a matter of weeks, after which they could move back to their homes.
\footcite[][84-85]{pappe}
As Samuel Katz concisely states, ``The Arab refugees were not driven from Palestine by anyone. The vast majority left, whether of their own free will or at the orders or exhortations of their leaders, always with same reassurance—that their departure would help in the war against Israel."
\footcite[][64]{dajani}



Even as the Hagana implemented the new Plan Dalet, the large majority of Palestinian exodus was not due to forced expulsion, but instead preceded confrontation with the Jews.  Of the confrontations and forced removals of Palestinians, Plan Dalet was often interpreted at the local level, allowing more extreme actions towards Palestinians.  Ultimately, however, there was no coherent plan on the part of Israel to uproot the existing Palestinian population.



The civil war that began the Arab Israeli War of 1948 began with Palestinian violence in reaction to the U.N. Partition plan, and ended with the Hagana's offensive strategy in the implementation of Plan Dalet.  While the Yishuv did take over territory outside that allotted to the formation of the Jewish state by the end of the war, which inherently led to a mass exodus of Palestinian Arabs, there was no predetermined attitude held by the Yishuv to accomplish this.  Instead, through a combination of largely unprovoked Palestinian evacuation and Hagana preparation for the impending foreign Arab invasion, the State of Israel formed with more territory and fewer Palestinian Arabs than intended.

%=================Section E====================

\section{Section E: Conclusion}
The Arab-Israeli War of 1948 was a continuation of conflict between the Palestinian Arab and the Jewish population.  The war resulted in the creation of the state of Israel and the mass departure of the Palestinians.  Some historians claim that the Yishuv intentionally implemented an aggressive offensive strategy to seize more Palestinian territory and expel the native Arab population.  However, the historical evidence points instead to the assertion that the Yishuv had no such intentions, and the mass exodus of the Arabs was a consequence of the poor Palestinian leadership, hostile domestic environments (with the assumption that the war would be quickly and easily won), and nature of the war itself.



\printbibliography[title=Section F: Bibliography]{}



\end{document}
