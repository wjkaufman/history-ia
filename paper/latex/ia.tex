\documentclass{turabian-researchpaper}

\usepackage[utf8]{inputenc}
\usepackage{turabian-formatting}
\usepackage[backend=biber]{biblatex-chicago}

\addbibresource{mybib.bib}

\begin{document}
\title{To what extent was the Yishuv's policy in the Arab Israeli War of 1948 one of territorial expansion and expulsion of the native Arab population?}

\author{Will Kaufman \\ Mr. Niedringhaus \\ IB World History}
\maketitle

\section{Section A}
In this investigation, several methodologies will be utilized to thoroughly examine the question of whether the Arab Israeli War of 1948 was a defensive war on the part of the Israelis.  The primary method of investigation will be the analysis of conflicting historiographies, generally taken from Israel's perspective and the Arabs' perspective.  In order to ensure a comprehensive analysis of these competing historiographies, a combination of primary and secondary sources sources will be used.  For secondary sources, the date of publication will be considered, as, in accordance with the “thirty-year rule” in both the United Kingdom and the state of Israel, secondary sources published before 1978 will not take into account information that was declassified after 1978.

\section{Section B}
The Arab-Israeli War of 1948 was divided into two different parts: the “civil war” fought between the Jewish state and the Palestinian Arabs, and the second phase of the war fought between the Jewish state and other Arab nations.  The first phase of the war began shortly after the United Nations (UN) General Assembly vote on Resolution 181 on 29 November 1947.  The UN Special Committee on Palestine (UNSCOP) worked from April to September of 1947 to research and identify a solution to the Jewish-Arab conflicts, and ultimately proposed the partition of the Palestinian territory to form two nations, one for the creation of a Jewish state and one for the Arab population.
\footcite[22]{pappe}
The partition resolution (Resolution 181) conflicted with the demands from the Arab Higher Committee and the Arab League, both of which opposed the partition plan and instead called for a single Palestinian state.
\footcite[][23]{pappe}
The Arab Higher Committee furthered its demands by requesting that Jewish immigration into Palestine be severely limited.  Despite the dissent from the various Arab representatives, and with the help of the Jewish Agency's far more diplomatic approach to the UNSCOP, the partition resolution passed.  The next day, on November 30, Palestinian Arabs began perpetrating sporadic acts of violence in urban centers as well as Jewish settlements.
\footcite[][77]{morris}
By December of 1947, acts of retaliation from both sides escalated the conflict into a civil war.

Before the start of the civil war, it was unclear whether the Arab Palestinians or the Yishuv (the Jewish community in Palestine before the creation of Israel) had the upper hand in battle.  The Palestinians outnumbered the Yishuv by a factor of two (1.2 million Palestinians as compared to 630,000 Jews), and the Palestinians controlled far more land and more of the highland areas than did the Jewish community.
\footcite[][30]{bartal}
However, the Hagana, the Jewish military organization, was far better equipped, better trained, and had more military experience than the Palestinian Arab militias and the Arab Liberation Army (ALA) that fought in the war.
\footcite[][81]{morris}
This question of military superiority caused other nations, including the United States, to abstain from supporting the creation of the Jewish state.

Initially, the war appeared to be going well for the Palestinians.  Fawzi al-Qawqji, field commander in the ALA, led several successful operations against Jewish convoys in January of 1948, and attacks on Jewish kibbutzim by Arab militias were initially successful.
\footcite[][79]{pappe}
By fighting small, localized conflicts, the Palestinians (with outside Arab support) were able to win several battles, hold territory between Jewish urban centers and roads, and force the Jewish leadership to question their military strategy.  The early success for Palestinians was short-lived, as new military policies effectively defended Jewish settlements and opened the roads for convoys to enter urban centers.  However, the Hagana could not completely defeat the Arab forces, only defend against Arab attacks and issue orders of retaliation against Arab aggression.
By mid-March of 1948, the inconclusive civil war between the Palestinians and the imminent Arab threat by May 15th forced the Jewish leadership to adopt an entirely different strategy.  David Ben Gurion, in accordance with members of the Hagana General Staff adopted a new plan, Plan Dalet, to transition to go on the offensive and prepare for the coming invasion of Arab troops.  Plan Dalet called for the securement of all territory outlined in the UN partition for the Jewish state, as well as territories bordering the future belligerent Arab states and any “enemy bases” in the Palestinian territory.
\footcite{khalidi}
By “enemy bases,” the plan implied any Palestinian villages that held Arab forces or that had been the base of a past enemy attack.
\footcite[][119]{morris}
Before implementing the offensive plan, the Hagana waited until the number of British troops diminished further, so as to meet with less resistance with the new offensive.  Furthermore, the Hagana received much needed Czech weapons and ammunition to fully implement Plan D.
\footcite[][116]{morris}

The first operation to implement Plan D was Operation Nahshon, in which Hagana troops went on the offensive from the Kibbutz Hulda to Jerusalem on April 4th 1948 in order to open the read to Jerusalem for supplies.
\footcite[][121]{morris}
Within only a few days, members of the Irgun and Lehi, underground Jewish organizations that fought against the British mandate before 1948, led a massacre on April 9th in the Arab village of Deir Yassin in which 100 men, women, and children were killed.
\footcite{nabka}
Until My 15th, when the surrounding Arab nations invaded and the second phase of the war began, the Hagana continued offensive operations in accordance with Plan Dalet.


\section{Section C}
\fullcite{pappe} ~\\
The Making of the Arab-Israeli Conflict, 1947-51 is a book written by Ilan Pappé, published in 1992.  The purpose of the book is to present the history of the Arab-Israeli War of 1948 and the ensuing tensions in the region with special emphasis on the changes in historiography due to the declassification of information in both British and Israeli archives.  The source provides valuable explication and commentary on the Arab-Israeli War, presents sources from all sides of the conflict, thus avoiding partisan bias on the part of Israel or the Arab nations.  However, the source is limited because the author Ilan Pappé is also a political activist in Israeli politics, so his choice of sources and analysis may be biased to support his own political beliefs, including economic and political boycotts against Israel and the formation of a binational state of Palestine.


\fullcite{nabka} ~\\
"The Nabka, 65 Years of Dispossession and Apartheid" is an article written by the Institute for Middle East Understanding in May of 2013.  The purpose of the website is to “to increase the public's understanding about the socio-economic, political and cultural aspects of Palestine, Palestinians and Palestinian Americans.”  The purpose of the article specifically is to present various facts that support the “ethnic cleansing” of the Palestinian territory in the late 1940's.  The source provides valuable perspectives and pieces of evidence from various revisionist historians, as well as information regarding the consequences of the events during the Arab Israeli War.  However, the source is limited because the Institute for Middle East Understanding is a pro-Palestinian organization and the article clearly omits facts that challenge the idea of the forced expulsion of Palestinians.

\section{Section D}

The Arab Israeli War of 1948 reflected and continued the conflict between the Jews and the surrounding Arabs, both within and outside of the Palestinian territory.  In the early stages of the war, nearly all perspectives concur on the Hagana's explicit defensive strategy against the Palestinian Arabs.  However, some sources indicate that the Yishuv changed strategies after the first months of war with the purpose of going on the offensive, expelling the Arabs, and ultimately controlling territory outside the allotted UN partition territory designated for Israel.  Other sources claim that the Hagana was forced to implement an offensive campaign in order to firmly establish transportation routes between urban centers, protect isolated Jewish settlements, and prepare for an impending invasion of Arab troops.  While the evidence available from the war corroborates the Hagana's change in strategy from defensive to offensive, it also indicates that the Hagana intended only to secure transportation routes, Jewish settlements, and prepare for other Arab countries to invade the territory.  Extreme terrorist groups (such as LHI and IZL ) did commit horrendous acts against Arab populations, but these acts were ultimately not the main driving force behind the mass exodus of Palestinian Arabs.

By the end of the year 1947, the UN partition resolution was released, and preliminary acts of violence were beginning in Palestine.  According to the Institute for Middle East Understanding, the Jewish leadership had no intention of following the UN partition plan and was instead planning to take over the Palestinian territory by force.
\footcite{nabka}
Jewish paramilitary groups such as IZL and LHI intensified their attacks against Arabs and British soldiers early on in the conflict, thus clearly displaying an unprovoked offensive strategy from the beginning of the war.
\footcite{nabka}
Already by December of 1947, only two months into the war, 75,000 Palestinian Arabs were displaced from their homes.

The statements above from the Institute for Middle East Understanding are not wholly accurate and do not reflect the entire situation during the war.  As claimed by Benny Morris, the Yishuv accepted the UN partition resolution and, for the first months of the war, held a purely defensive position against Arab attacks.
\footcite[][98-100]{morris}
Acts of Arab aggression were met with limited retaliation on the part of the Hagana, although other groups such as IZL and LHI were too difficult and extremist to control.  And while it is true that tens of thousands of Palestinians had left their homes by the end of 1947, it was less due to Israeli aggression than a host of other factors: most of the Palestinian leaders had fled the territory in late 1947, leaving the Palestinian people without leadership or organization; the Arab economy was debilitated from the start of the conflict, in part from the lack of organization and preparation surrounding economic policy; and the Palestinian Arabs, including most other people on both sides of the conflict, initially believed that the conflict would only last a matter of weeks, after which they could move back to their homes.
\footcite[][84-85]{pappe}
Even when the Hagana implemented the new Plan Dalet, the large majority of Palestinian exodus was not due to forced expulsion, but instead preceded confrontation with the Jews.  Of the confrontations and forced removals of Palestinians, Plan Dalet was often interpreted at the local level, allowing more extreme actions towards Palestinians.  Ultimately, however, there was no coherent plan on the part of Israel to uproot the existing Palestinian population.

The civil war that began the Arab Israeli War of 1948 began with Palestinian violence in reaction to the U.N. Partition plan, and ended with the Hagana's offensive strategy in the implementation of Plan Dalet.  While the Yishuv did take over territory outside that allotted to the formation of the Jewish state by the end of the war, which inherently led to a mass exodus of Palestinian Arabs, there was no predetermined attitude held by the Yishuv to accomplish this.  Instead, through a combination of largely unprovoked Palestinian evacuation and Hagana preparation for the impending foreign Arab invasion, the State of Israel formed with more territory and fewer Palestinian Arabs than intended.

\section{Section E}


\printbibliography{}

\end{document}
