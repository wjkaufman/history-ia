\documentclass{turabian-researchpaper}

\usepackage[utf8]{inputenc}
\usepackage{turabian-formatting}
\usepackage[backend=biber]{biblatex-chicago}

\addbibresource{mybib.bib}

\begin{document}
\title{To what extent was the Yishuv's policy in the Arab Israeli War of 1948 one of territorial expansion and expulsion of the native Arab population?}

\author{Will Kaufman \\ Mr. Niedringhaus}
\maketitle

\section{Section A}
In this investigation, several methodologies will be utilized to thoroughly examine the question of whether the Arab Israeli War of 1948 was a defensive war on the part of the Israelis.  The primary method of investigation will be the analysis of conflicting historiographies, generally taken from Israel's perspective and the Arabs' perspective.  In order to ensure a comprehensive analysis of these competing historiographies, a combination of primary and secondary sources sources will be used.  For secondary sources, the date of publication will be considered, as, in accordance with the “thirty-year rule” in both the United Kingdom and the state of Israel, secondary sources published before 1978 will not take into account information that was declassified after 1978.

\section{Section B}
The Arab-Israeli War of 1948 was divided into two different parts: the “civil war” fought between the Jewish state and the Palestinian Arabs, and the second phase of the war fought between the Jewish state and other Arab nations.  The first phase of the war began shortly after the United Nations (UN) General Assembly vote on Resolution 181 on 29 November 1947.  The UN Special Committee on Palestine (UNSCOP) worked from April to September of 1947 to research and identify a solution to the Jewish-Arab conflicts, and ultimately proposed the partition of the Palestinian territory to form two nations, one for the creation of a Jewish state and one for the Arab population.
\footfullcite[22]{pappe}
The partition resolution (Resolution 181) conflicted with the demands from the Arab Higher Committee and the Arab League, both of which opposed the partition plan and instead called for a single Palestinian state.
\footnote{Ibid, 23.}
The Arab Higher Committee furthered its demands by requesting that Jewish immigration into Palestine be severely limited.  Despite the dissent from the various Arab representatives, and with the help of the Jewish Agency's far more diplomatic approach to the UNSCOP, the partition resolution passed.  The next day, on November 30, Palestinian Arabs began perpetrating sporadic acts of violence in urban centers as well as Jewish settlements.
\footnote{Benny Morris, 1948: A History of the First Arab-Israeli War (New Haven: Yale University Press, 2008), 77.}
By December of 1947, acts of retaliation from both sides escalated the conflict into a civil war.

Before the start of the civil war, it was unclear whether the Arab Palestinians or the Yishuv (the Jewish community in Palestine before the creation of Israel) had the upper hand in battle.  The Palestinians outnumbered the Yishuv by a factor of two (1.2 million Palestinians as compared to 630,000 Jews), and the Palestinians controlled far more land and more of the highland areas than did the Jewish community4.  However, the Hagana, the Jewish military organization, was far better equipped, better trained, and had more military experience than the Palestinian Arab militias and the Arab Liberation Army (ALA) that fought in the war5.  This question of military superiority caused other nations, including the United States, to abstain from supporting the creation of the Jewish state.

Initially, the war appeared to be going well for the Palestinians.  Fawzi al-Qawqji, field commander in the ALA, led several successful operations against Jewish convoys in January of 1948, and attacks on Jewish kibbutzim by Arab militias were initially successful6.  By fighting small, localized conflicts, the Palestinians (with outside Arab support) were able to win several battles, hold territory between Jewish urban centers and roads, and force the Jewish leadership to question their military strategy.  The early success for Palestinians was short-lived, as new military policies effectively defended Jewish settlements and opened the roads for convoys to enter urban centers.  However, the Hagana could not completely defeat the Arab forces, only defend against Arab attacks and issue orders of retaliation against Arab aggression.
By mid-March of 1948, the inconclusive civil war between the Palestinians and the imminent Arab threat by May 15th forced the Jewish leadership to adopt an entirely different strategy.  David Ben Gurion, in accordance with members of the Hagana General Staff adopted a new plan, Plan Dalet, to transition to go on the offensive and prepare for the coming invasion of Arab troops.  Plan Dalet called for the securement of all territory outlined in the UN partition for the Jewish state, as well as territories bordering the future belligerent Arab states and any “enemy bases” in the Palestinian territory7.  By “enemy bases,” the plan implied any Palestinian villages that held Arab forces or that had been the base of a past enemy attack8.  Before implementing the offensive plan, the Hagana waited until the number of British troops diminished further, so as to meet with less resistance with the new offensive.  Furthermore, the Hagana received much needed Czech weapons and ammunition to fully implement Plan D9.

The first operation to implement Plan D was Operation Nahshon, in which Hagana troops went on the offensive from the Kibbutz Hulda to Jerusalem on April 4th 1948 in order to open the read to Jerusalem for supplies10.  Within only a few days, members of the Irgun and Lehi, underground Jewish organizations that fought against the British mandate before 1948, led a massacre on April 9th in the Arab village of Deir Yassin in which 100 men, women, and children were killed11.  Until My 15th, when the surrounding Arab nations invaded and the second phase of the war began, the Hagana continued offensive operations in accordance with Plan Dalet.

%\printbibliography{}

Help please

\end{document}
